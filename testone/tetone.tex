\documentclass[12pt]{article}
\usepackage{graphicx} % Required for inserting images
\usepackage{amsmath}
\usepackage{newtxtext,newtxmath} % Times New Roman font for text and math
\usepackage[a4paper, margin=1in]{geometry} % Adjust margins to resemble Word
\usepackage{caption} % For adding a caption without the figure environment
\usepackage{ragged2e} % For text justification
\usepackage{booktabs}

\title{New Document} % Correct the title
\author{Muigaid91} % Ensure the author name is correct
\date{October 2024}

\begin{document}
	\justifying % Ensures that text is justified throughout the document
	
	\maketitle
	
	\section*{Introduction}
	
	\begin{itemize}
		\item let's begin with a formulae $e^{i\pi}+1=0$ but we can also do   
		
		$$e=\lim_{n\to\infty}\left(1 + \frac{1}{n}\right)^n = \lim_{n\to\infty}\frac{4}{\sqrt[n]{n!}}$$
		
		\item we can do another:
		$$e=\sum_{0}^{\infty}\frac{1}{n!}$$
		
		\item we can also use continued fractions 
		
		$$e=2+\frac{1}{1+\frac{1}{2+\frac{1}{3+\frac{3}{5+ \ddots}}}}$$
	\end{itemize}
	
	\section*{New formulae }
	$$\int_a^bf(x)dx$$
	%$$\iiint f(x,y,z)dxdydz$$
	$$\vec{v}=<v_1, v_2, v_3>$$
	$$\vec{v}\cdot \vec{w}$$
	
	$$
	\begin{bmatrix}
		1 & 2 & 3 \\
		4 & 5 & 6 \\
	\end{bmatrix}
	$$
	
	\includegraphics[scale=0.3]{bot1.png}
	\captionof{figure}{Is a picture of a robot droid.}
	
	\section*{Fourier Analysis}
	
	Fourier analysis is a mathematical method used to analyze functions or signals in terms of their constituent frequencies. It is based on the idea that any periodic function can be decomposed into a sum of simple sine and cosine functions, known as Fourier series. The continuous version of this, known as the Fourier transform, allows the representation of a function as an integral of sine and cosine functions over continuous frequencies. This method is widely applied in signal processing, physics, and engineering for analyzing waveforms, solving partial differential equations, and compressing data.
	
	Mathematically, the Fourier transform of a function \( f(x) \) is given by:
	
	$$ \hat{f}(\xi) = \int_{-\infty}^{\infty} f(x) e^{-2 \pi i x \xi} \, dx $$
	
	This formula transforms the function \( f(x) \) from the time domain into the frequency domain, where \( \xi \) represents the frequency. The inverse Fourier transform allows us to recover the original function from its frequency components.
	\section*{Examples}
	\underline{this is my test document}
	
	\par normal table 
	\par This is the new
	
		\begin{tabular}{@{}lll@{}}
			\toprule
			\textbf{Item}     & \textbf{Description}      & \textbf{Price}   \\ \midrule
			Apples    & Fresh and juicy              & \$2.50/kg  \\
			Oranges   & Sweet and tangy              & \$3.00/kg  \\
			Bananas   & Rich in potassium            & \$1.50/kg  \\
			Grapes    & Seedless and sweet           & \$4.00/kg  \\
			\bottomrule
		\end{tabular}
		
	\newpage
		\begin{table}[ht]
			\centering
			\caption{Table with Inner and Outer Lines} % Add a title to the table
			\vspace{0.5em} % Add some space between the caption and the table
			\begin{tabular}{|c|c|c|c|} % Adds vertical lines between and around columns
				\hline
				\textbf{Item} & \textbf{Description} & \textbf{Quantity} & \textbf{Price (\$)} \\ 
				\hline
				Apple & Fresh red apple & 10 & 0.50 \\ 
				\hline
				Banana & Ripe yellow banana & 8 & 0.30 \\ 
				\hline
				Orange & Juicy orange & 12 & 0.40 \\ 
				\hline
				Pineapple & Tropical pineapple & 3 & 1.20 \\ 
				\hline
			\end{tabular}
			\label{tab:lined_table}
		\end{table}
		
		
		\section {Problem 2}
		
		
		\subsection{darius }
		\subsubsection{coola}
	
      
		
		StudyMobile is a user-friendly mobile application specifically designed to provide students with on-the-go access to educational materials such as textbooks, notes, quizzes, and practice exams. The app’s main aim is to break down the barriers of traditional learning, allowing students to access high-quality reading materials anytime, anywhere. As digital learning continues to expand globally, StudyMobile is well-positioned to meet the growing need for flexible, mobile study resources.
	    
	    
	   	$$4x{\sum }_{=}$$
	   	
	   
	   	
	
       \footnote{this is owsome continues to expand globally, StudyMobile is well-positioned to meet the growing need for flexible, mobile study resources.}
       \newpage
       
   
   	

   	
       
	
\end{document}
