
\documentclass[12pt]{article}
\usepackage{graphicx} % Required for inserting images
\usepackage{amsmath}
\usepackage{newtxtext,newtxmath} % Times New Roman font for text and math
\usepackage[a4paper, margin=1in]{geometry} % Adjust margins to resemble Word
\usepackage{caption} % For adding a caption without the figure environment
\usepackage{ragged2e} % For text justification
\usepackage{booktabs}
\usepackage{multicol}


\title{CAT 1: Transient Analysis of RL, RC, and RLC Circuits} % Correct the title
\author{David Muigai} % Ensure the author name is correct




\begin{document}

\maketitle


Determine the dimensions of the core, the number of turns, the cross-section area of conductors in primary and secondary windings of a 100 kVA, 2200/480 V, 1-phase, core type transformer, to operate at a frequency of 50Hz, by assuming the following data. Approximate Volt per turn = 7.5 Volt. Maximum flux density = 1.2 Wb/m\textsuperscript{2}. Ratio of effective cross-sectional area of core to square of diameter of circumscribing circle is 0.6. Ratio of height to width of window is 2. Window space factor = 0.28. Current density = 2.5 A/mm\textsuperscript{2}.

\subsection*{Given Data}
\begin{itemize}
	\item 100 kVA, 50 Hz
	\item 2200/480 V
	\item $E_t = 7.5$ V
	\item $K_w = 0.28$
	\item $A_e/d^2 = 0.6$
	\item $B_m = 1.2$ Wb/m\textsuperscript{2}
	\item $H/W = 2$
	\item Single-phase
	\item Current density, $\delta = 2.5$ A/mm\textsuperscript{2}
\end{itemize}


\section*{Step 1: Calculate the number of turns in primary and secondary windings}

\subsection*{Number of primary turns ($N_1$)}
\[
N_1 = \frac{V_1}{E_t} = \frac{2200}{7.5} = 293.33 \approx 294 \text{ turns}
\]

\subsection*{Number of secondary turns ($N_2$)}
\[
N_2 = \frac{V_2}{E_t} = \frac{480}{7.5} = 64 \text{ turns}
\]

\section*{Step 2: Calculate the maximum value of flux ($\Phi_m$)}
From the equation: 
\[
E_t = 4.44 \times f \times \Phi_m
\]
\[
\Phi_m = \frac{E_t}{4.44 \times f} = \frac{7.5}{4.44 \times 50} = 0.0338 \text{ Wb}
\]

\section*{Step 3: Calculate the cross-sectional area of the core ($A_e$)}
Since 
\[
\Phi_m = B_m \times A_e
\]
\[
A_e = \frac{\Phi_m}{B_m} = \frac{0.0338}{1.2} = 0.0282 \text{ m}^2
\]

\section*{Step 4: Calculate the diameter of the circumscribing circle ($d$)}
Given that 
\[
\frac{A_e}{d^2} = 0.6
\]
\[
d^2 = \frac{A_e}{0.6} = \frac{0.0282}{0.6} = 0.047 \text{ m}^2
\]
\[
d = \sqrt{0.047} = 0.2168 \text{ m} = 216.8 \text{ mm}
\]

\section*{Step 5: Calculate the primary and secondary currents}
\subsection*{Primary current ($I_1$)}
\[
I_1 = \frac{\text{kVA} \times 1000}{V_1} = \frac{100 \times 1000}{2200} = 45.45 \text{ A}
\]

\subsection*{Secondary current ($I_2$)}
\[
I_2 = \frac{\text{kVA} \times 1000}{V_2} = \frac{100 \times 1000}{480} = 208.33 \text{ A}
\]

\section*{Step 6: Calculate the cross-sectional area of primary and secondary conductors}
\subsection*{Primary conductor area ($a_1$)}
\[
a_1 = \frac{I_1}{\delta} = \frac{45.45}{2.5} = 18.18 \text{ mm}^2
\]

\subsection*{Secondary conductor area ($a_2$)}
\[
a_2 = \frac{I_2}{\delta} = \frac{208.33}{2.5} = 83.33 \text{ mm}^2
\]

\section*{Step 7: Calculate the window dimensions}
\subsection*{Total copper area in the window ($A_{cu}$)}
\[
A_{cu} = (N_1 \times a_1 + N_2 \times a_2)
\]
\[
A_{cu} = (294 \times 18.18 + 64 \times 83.33) = 5344.9 + 5333.1 = 10678 \text{ mm}^2
\]

\subsection*{Window area ($A_w$)}
\[
A_w = \frac{A_{cu}}{K} = \frac{10678}{0.28} = 38135.7 \text{ mm}^2 = 0.0381 \text{ m}^2
\]

\subsection*{Window dimensions}
Given $H/W = 2$, where $H$ is the height and $W$ is the width of the window:
\[
A_w = H \times W = 2W \times W = 2W^2
\]
\[
2W^2 = 0.0381
\]
\[
W^2 = 0.01905
\]
\[
W = 0.138 \text{ m} = 138 \text{ mm}
\]

\subsection*{Height of window ($H$)}
\[
H = 2W = 2 \times 138 = 276 \text{ mm}
\]

\section*{Summary of Results}

\subsection*{Core dimensions}
\begin{itemize}
	\item Cross-sectional area ($A_e$) = 0.0282 m\textsuperscript{2}
	\item Diameter of circumscribing circle ($d$) = 216.8 mm
\end{itemize}

\subsection*{Number of turns}
\begin{itemize}
	\item Primary winding ($N_1$) = 294 turns
	\item Secondary winding ($N_2$) = 64 turns
\end{itemize}

\subsection*{Cross-sectional area of conductors}
\begin{itemize}
	\item Primary winding ($a_1$) = 18.18 mm\textsuperscript{2}
	\item Secondary winding ($a_2$) = 83.33 mm\textsuperscript{2}
\end{itemize}

\subsection*{Window dimensions}
\begin{itemize}
	\item Width ($W$) = 138 mm
	\item Height ($H$) = 276 mm
\end{itemize}



\section*{Example 2} 
Calculate the dimension of the core, the number of turns, and the cross-sectional area of conductors in the primary and secondary windings of a 100 kVA, 2300/400V, 50Hz, 1-phase, shell-type transformer. The given data is as follows:

\subsection*{Given Data}
\begin{itemize}
	\item Apparent power: $100$ kVA
	\item Voltage: $2300/400$ V
	\item Frequency: $50$ Hz
	\item Transformer type: Single-phase, shell-type
	\item Ratio of magnetic and electric loading: $480 \times 10^{-8}$
	\item Maximum flux density: $B_m = 1.1$ Wb/m$^2$
	\item Current density: $\delta = 2.2$ A/mm$^2$
	\item Window space factor: $K_w = 0.3$
	\item Stacking factor: $S_f = 0.9$
	\item Depth of core / Width of central limb = $2.6$
	\item Height of window / Width of window = $2.5$
\end{itemize}



\section*{Step 1: Core Area Calculation}
The core area $A_c$ is calculated using the formula:

\begin{equation}
	A_c = \frac{S}{4.44 \times f \times B_m \times N}
\end{equation}

Given:

\begin{align*}
	S &= 100,000 \text{ VA} \\
	f &= 50 \text{ Hz} \\
	B_m &= 1.1 \text{ Wb/m}^2 \\
	N &= 480 \times 10^3
\end{align*}

\begin{equation}
	A_c = \frac{100,000}{4.44 \times 50 \times 1.1 \times 480,000} = \frac{100,000}{117,216,000} \approx 0.0423 \text{ m}^2
\end{equation}

\section*{Step 2: Core Cross-Section Dimensions}
Using the stacking factor (0.9):

\begin{align*}
	\text{Core Width} &= \frac{A_c}{d} = \frac{0.0423}{2.6} \approx 0.1533 \text{ m} \\
	\text{Core Depth} &= d = 0.39859 \text{ m}
\end{align*}

\section*{Step 3: Window Area Calculation}

\begin{equation}
	A_w = H_w \times W_w = 2.5 \times 0.1083 = 0.2708 \text{ m}^2
\end{equation}

\section*{Step 4: Number of Turns (Primary and Secondary)}

\begin{equation}
	T_p = \frac{V_1}{4.44 \times f \times B_m \times A_c} = \frac{2300}{4.44 \times 50 \times 1.1 \times 0.0423} = \frac{2300}{103.1583} \approx 223 \text{ turns}
\end{equation}

Similarly, for the secondary:

\begin{equation}
	T_s = \frac{400}{4.44 \times 50 \times 1.1 \times 0.0423} = \frac{400}{103.1583} \approx 39 \text{ turns}
\end{equation}

\section*{Step 5: Cross-sectional Area of Conductors}
Using current density:

\begin{equation}
	A_g = \frac{I}{k}
\end{equation}

For primary current:

\begin{align*}
	I_p &= \frac{100,000}{2300} \approx 43.48 \text{ A} \\
	A_g &= \frac{43.48}{2.2} \approx 19.76 \text{ mm}^2
\end{align*}

For secondary current:

\begin{align*}
	I_s &= \frac{100,000}{400} = 250 \text{ A} \\
	A_s &= \frac{250}{2.2} \approx 113.64 \text{ mm}^2
\end{align*}






\end{document}
