\documentclass[12pt]{article}
\usepackage{graphicx} % Required for inserting images
\usepackage{amsmath}
\usepackage{newtxtext,newtxmath} % Times New Roman font
\usepackage[a4paper, margin=1in]{geometry} % Standard margins
\usepackage{ragged2e} % For text justification
\usepackage{titlesec} % For title formatting
\usepackage{setspace} % Line spacing
\usepackage{array} % Table alignment
\usepackage{graphicx}




\begin{document}
	
		
	\begin{center}
		\textbf{CHUKA} \hspace{2cm}
		\includegraphics[width=0.3\textwidth]{images/picture1} % Replace "logo.png" with your university logo file
		\hspace{2cm} \textbf{UNIVERSITY} \\
		\vspace{0.3cm}
		\textbf{UNIVERSITY EXAMINATIONS} \\
		\vspace{0.3cm}
		\textbf{EXAMINATION FOR THE AWARD OF DEGREE OF BACHELOR OF SCIENCE IN} \\
		\textbf{ELECTRICAL AND ELECTRONICS ENGINEERING} \\
		\vspace{0.3cm}
		\textbf{EENG 475: CONTROL ENGINEERING 1} \\
		\vspace{0.3cm}
		\textbf{STREAMS: \hspace{5cm} TIME: 2 HOURS} \\
		\vspace{0.3cm}
		\textbf{DAY/DATE: TUESDAY 15/04/2025 \hspace{2cm} 8.30 A.M – 10.30 A.M} \\
		\rule{\textwidth}{0.4pt} % horizontal line
	\end{center}
	
	\vspace{0.5cm}
	
	\noindent \textbf{INSTRUCTIONS} \\
	Answer question ONE and any other TWO questions.\\
	Do not write on the question paper.
	
	\vspace{0.5cm}
	
	\noindent \textbf{QUESTION ONE (30 MARKS)}
	\begin{enumerate}
		\item[(a)] Define the following terms: 
		\begin{itemize}
			\item \textbf{State Variables:} State variables are a set of variables that describe the smallest possible subset of system variables such that the knowledge of these variables at any time \( t_0 \), together with the input for \( t \geq t_0 \), completely determines the behavior of the system for all future times \( t \geq t_0 \).
			
			\item \textbf{State Vector:} A state vector is a column vector that contains all the state variables of a system. It represents the complete state of the system at a given time and is typically denoted as \( \mathbf{x}(t) \).
			
			\item \textbf{State Space:} State space is a mathematical model of a physical system expressed as a set of input, output, and state variables related by first-order differential (or difference) equations. It provides a framework to model and analyze systems using a set of equations in matrix form.
		\end{itemize}
		
		\hfill [3 marks]
		
		
		
		
		
		
		\item[(b)] What are the key advantages of state-space representation over classical control techniques? List and briefly explain three advantages. \hfill [3 marks]
		
		\item[(c)] Define the following terms in the context of digital control systems:
		\begin{itemize}
			\item Sampling
			\item Zero-order hold
			\item Quantization
		\end{itemize}
		\hfill [2 marks]
		
		\item[(d)] What are eigenvalues and eigenvectors, and how are they used in state-space analysis? \hfill [3 marks]
		
		\item[(e)] Define controllability and observability. Provide the criteria for testing them. \hfill [3 marks]
		
		\item[(f)] Describe the difference between continuous-time and discrete-time systems in control. \hfill [2 marks]
	\end{enumerate}
	
	
	
	\section*{Question Two \hfill (20 Marks)}
	
	\begin{enumerate}
		\item[(a)] \subsubsection*{Test the controllability and observability of the system below:  [4 marks]} 
		\[
		A = \begin{bmatrix} 1 & 0 \\ 0 & 2 \end{bmatrix}, \quad
		B = \begin{bmatrix} 2 \\ 0 \end{bmatrix}, \quad
		C = \begin{bmatrix} 1 & 1 \end{bmatrix}
		\]
		
		\textbf{Solution:} \\
		The controllability matrix is:
		\[
		\mathcal{C} = [B \quad AB] = \begin{bmatrix} 2 & 2 \\ 0 & 0 \end{bmatrix}
		\]
		Since \(\text{rank}(\mathcal{C}) = 1 < 2\), the system is \textbf{not controllable}. \\
		
		The observability matrix is:
		\[
		\mathcal{O} = \begin{bmatrix} C \\ CA \end{bmatrix} = \begin{bmatrix} 1 & 1 \\ 1 & 2 \end{bmatrix}
		\]
		\[
		\det(\mathcal{O}) = (1)(2) - (1)(1) = 1 \neq 0
		\]
		Therefore, the system is \textbf{observable}.
		
		\item[(b)] \textbf{For the state-space system:}
		\[
		\dot{x} = \begin{bmatrix} 2 & 1 \\ 0 & -3 \end{bmatrix}x + \begin{bmatrix} 1 \\ 2 \end{bmatrix}u
		\quad , \quad
		y = \begin{bmatrix} 4 & 5 \end{bmatrix}x
		\]
		
		\begin{enumerate}
			\item[(i)] \textbf{Calculate the state transition matrix \( \Phi(t) \).} \hfill [4 marks]
			
			\textbf{Solution:} \\
			
			
			
			\subsection*{(i) State Transition Matrix \(\Phi(t)\)}
			
			The state transition matrix is:
			
			\[
			\Phi(t) = e^{At}
			\quad \text{where} \quad A = 
			\begin{bmatrix}
				2 & 1 \\ 
				0 & -3
			\end{bmatrix}
			\]
			
			Since \(A\) is upper triangular, the eigenvalues are \(2\) and \(-3\).
			
			Thus:
			
			\[
			\Phi(t) = 
			\begin{bmatrix}
				e^{2t} & f(t) \\ 
				0 & e^{-3t}
			\end{bmatrix}
			\]
			
			where \(f(t)\) satisfies:
			
			\[
			\frac{d}{dt} f(t) = 1 \times e^{-3t} + 2f(t)
			\]
			
			Using integrating factor \(e^{-2t}\):
			
			\[
			\frac{d}{dt}\left( e^{-2t} f(t) \right) = e^{-5t}
			\]
			
			Integrating:
			
			\[
			e^{-2t} f(t) = \frac{e^{-5t}}{-5} + C
			\]
			\[
			f(t) = e^{2t} \left( \frac{1}{5}(1 - e^{-5t}) \right)
			\]
			
			Thus:
			
			\[
			\boxed{
				\Phi(t) = 
				\begin{bmatrix}
					e^{2t} & \frac{1}{5}e^{2t}(1 - e^{-5t}) \\ 
					0 & e^{-3t}
				\end{bmatrix}
			}
			\]
			
			
			\item[(ii)] \textbf{Find the zero-input response for the initial condition \( x(0) = \begin{bmatrix} 1 \\ -1 \end{bmatrix}^T \).} \hfill [3 marks]
			
			\textbf{Solution:} \\
			The zero-input response is:
			\[
			x(t) = \Phi(t) x(0) = 
			\begin{bmatrix}
				\frac{4}{5}e^{2t} + \frac{1}{5}e^{-3t} \\
				-e^{-3t}
			\end{bmatrix}
			\]
		\end{enumerate}
		
		\item[(c)] \textbf{For the continuous-time system:}
		\[
		\dot{x} = \begin{bmatrix} 0 & 1 \\ -2 & -3 \end{bmatrix}x + \begin{bmatrix} 0 \\ 1 \end{bmatrix}u
		\quad , \quad
		y = \begin{bmatrix} 1 & 0 \end{bmatrix}x
		\]
		
		\textbf{Design a state feedback controller \( u = -Kx \) to place the eigenvalues at \( s = -1 \) and \( s = -2 \).} \hfill [7 marks]
		
		\item[(d)] \textbf{Explain the concepts of reliability and redundancy in digital control systems.} \hfill [2 marks]
	\end{enumerate}
	
	
	
	
	
	
	
	
	
	
	\section*{Question b}
	
	For the state-space system:
	\[
	\dot{x} = 
	\begin{bmatrix}
		2 & 1 \\ 
		0 & -3
	\end{bmatrix} x + 
	\begin{bmatrix}
		1 \\ 
		2
	\end{bmatrix} u,
	\quad
	y = 
	\begin{bmatrix}
		4 & 5
	\end{bmatrix} x
	\]
	\begin{enumerate}
		\item[(i)] Calculate the state transition matrix \(\Phi(t)\).
		\item[(ii)] Find the zero-input response for the initial condition \(x(0) = \begin{bmatrix} 1 \\ -1 \end{bmatrix}\).
	\end{enumerate}
	
	\section*{solution}
	
	\subsection*{(i) State Transition Matrix \(\Phi(t)\)}
	
	The state transition matrix is:
	
	\[
	\Phi(t) = e^{At}
	\quad \text{where} \quad A = 
	\begin{bmatrix}
		2 & 1 \\ 
		0 & -3
	\end{bmatrix}
	\]
	
	Since \(A\) is upper triangular, the eigenvalues are \(2\) and \(-3\).
	
	Thus:
	
	\[
	\Phi(t) = 
	\begin{bmatrix}
		e^{2t} & f(t) \\ 
		0 & e^{-3t}
	\end{bmatrix}
	\]
	
	where \(f(t)\) satisfies:
	
	\[
	\frac{d}{dt} f(t) = 1 \times e^{-3t} + 2f(t)
	\]
	
	Using integrating factor \(e^{-2t}\):
	
	\[
	\frac{d}{dt}\left( e^{-2t} f(t) \right) = e^{-5t}
	\]
	
	Integrating:
	
	\[
	e^{-2t} f(t) = \frac{e^{-5t}}{-5} + C
	\]
	\[
	f(t) = e^{2t} \left( \frac{1}{5}(1 - e^{-5t}) \right)
	\]
	
	Thus:
	
	\[
	\boxed{
		\Phi(t) = 
		\begin{bmatrix}
			e^{2t} & \frac{1}{5}e^{2t}(1 - e^{-5t}) \\ 
			0 & e^{-3t}
		\end{bmatrix}
	}
	\]
	
	\subsection*{(ii) Zero-input Response}
	
	The zero-input response is:
	
	\[
	x(t) = \Phi(t) x(0)
	\]
	
	Substituting:
	
	\[
	x(t) = 
	\begin{bmatrix}
		e^{2t} & \frac{1}{5}e^{2t}(1 - e^{-5t}) \\ 
		0 & e^{-3t}
	\end{bmatrix}
	\begin{bmatrix}
		1 \\ -1
	\end{bmatrix}
	\]
	
	First component:
	
	\[
	e^{2t}(1) + \frac{1}{5}e^{2t}(1 - e^{-5t})(-1) = e^{2t} \left( \frac{4}{5} + \frac{1}{5}e^{-5t} \right)
	\]
	
	Second component:
	
	\[
	0(1) + e^{-3t}(-1) = -e^{-3t}
	\]
	
	Thus:
	
	\[
	\boxed{
		x(t) = 
		\begin{bmatrix}
			e^{2t} \left( \frac{4}{5} + \frac{1}{5}e^{-5t} \right) \\
			-e^{-3t}
		\end{bmatrix}
	}
	\]
	
	\newpage
	
	\section*{Question c}
	
	For the continuous-time system:
	
	\[
	\dot{x} = 
	\begin{bmatrix}
		0 & 1 \\ 
		-2 & -3
	\end{bmatrix} x +
	\begin{bmatrix}
		0 \\ 
		1
	\end{bmatrix} u,
	\quad
	y = 
	\begin{bmatrix}
		1 & 0
	\end{bmatrix} x
	\]
	
	Design a state feedback controller \(u = -Kx\) to place the eigenvalues at \(s = -1\) and \(s = -2\).
	
	\section*{solution}
	
	The closed-loop system is:
	
	\[
	\dot{x} = (A - BK)x
	\]
	
	where:
	
	\[
	B = 
	\begin{bmatrix}
		0 \\ 1
	\end{bmatrix},
	\quad
	K = 
	\begin{bmatrix}
		k_1 & k_2
	\end{bmatrix}
	\]
	
	Thus:
	
	\[
	A - BK =
	\begin{bmatrix}
		0 & 1 \\
		-2 - k_1 & -3 - k_2
	\end{bmatrix}
	\]
	
	The characteristic equation:
	
	\[
	\det(sI - (A - BK)) = 0
	\]
	
	Expanding:
	
	\[
	\det\left(
	\begin{bmatrix}
		s & -1 \\ 
		2+k_1 & s+3+k_2
	\end{bmatrix}
	\right) = 0
	\]
	
	Thus:
	
	\[
	s^2 + (3+k_2)s + (2+k_1) = 0
	\]
	
	The desired characteristic equation is:
	
	\[
	(s+1)(s+2) = s^2 + 3s + 2
	\]
	
	Comparing:
	
	\[
	3 + k_2 = 3 \quad \Rightarrow \quad k_2 = 0
	\]
	\[
	2 + k_1 = 2 \quad \Rightarrow \quad k_1 = 0
	\]
	
	Thus:
	
	\[
	\boxed{K = \begin{bmatrix} 0 & 0 \end{bmatrix}}
	\]
	
	
	
	
	
	
	\section*{Question Three \hfill (20 Marks)}
	
	\begin{enumerate}
		\item[(a)] Convert the continuous transfer function 
		\[
		G(s) = \frac{1}{s+2}
		\]
		to z-domain using \( T = 0.1 \, \text{seconds} \). \hfill [3 marks]
		
		\item[(b)] Convert the following transfer function to state-space representation in controllable canonical form:
		\[
		G(s) = \frac{5s + 3}{s^2 + 4s - 0.5}
		\]
		\hfill [3 marks]
		
		\item[(c)] Consider a continuous-time system:
		\[
		\dot{x} = 
		\begin{bmatrix}
			0 & 1 \\
			-2 & -3
		\end{bmatrix}x +
		\begin{bmatrix}
			0 \\ 1
		\end{bmatrix}u
		\quad , \quad
		y = 
		\begin{bmatrix}
			1 & 0
		\end{bmatrix}x
		\]
		
		\begin{enumerate}
			\item[(i)] Convert this system to a discrete-time system with a sampling period of \( T = 0.1 \, \text{seconds} \). \hfill [4 marks]
			
			\item[(ii)] Determine whether the discrete-time system is stable. \hfill [1 mark]
		\end{enumerate}
		
		\item[(d)] Consider the discrete-time system:
		\[
		x(k+1) = 
		\begin{bmatrix}
			0.5 & 0 \\
			0 & 0.8
		\end{bmatrix}x(k) +
		\begin{bmatrix}
			1 \\ 1
		\end{bmatrix}u(k)
		\quad , \quad
		y(k) = 
		\begin{bmatrix}
			1 & 1
		\end{bmatrix}x(k)
		\]
		
		\begin{enumerate}
			\item[(i)] Design a state feedback controller \( u(k) = -Kx(k) \) to place the eigenvalues of the closed-loop system at \( z = 0.3 \) and \( z = 0.4 \). \hfill [6 marks]
			
			\item[(ii)] Verify that the designed controller achieves the desired eigenvalue placement. \hfill [3 marks]
		\end{enumerate}
	\end{enumerate}
	
	
	
	
	\section*{Question Four \hfill (20 Marks)}
	
	\begin{enumerate}
		\item[(a)] Consider a temperature control system for an industrial furnace. The continuous-time plant model is:
		\[
		G(s) = \frac{2}{5s+1}
		\]
		You need to implement a digital controller with a sampling time of \( T = 0.5 \) seconds.
		
		\begin{enumerate}
			\item[(i)] Find the discrete-time plant model \( G(z) \). \hfill [4 marks]
			
			\item[(ii)] Design a digital PID controller in the form:
			\[
			C(z) = K_p + K_i\left( \frac{Tz}{z-1} \right) + \frac{K_d(z-1)}{Tz}
			\]
			with parameters \( K_p = 1.2 \), \( K_i = 0.5 \), and \( K_d = 0.1 \). \hfill [4 marks]
			
			\item[(iii)] Write the difference equation for the controller implementation. \hfill [3 marks]
		\end{enumerate}
		
		\item[(b)] A discrete-time control system has the following state-space representation:
		
		\[
		x(k+1) = 
		\begin{bmatrix}
			0.8 & 0.1 \\
			0 & 0.9
		\end{bmatrix}x(k) +
		\begin{bmatrix}
			0 \\ 1
		\end{bmatrix}u(k)
		\quad , \quad
		y(k) = 
		\begin{bmatrix}
			1 & 0
		\end{bmatrix}x(k)
		\]
		
		\begin{enumerate}
			\item[(i)] Design a full-order observer to estimate the states with observer poles at \( z = 0.2 \) and \( z = 0.3 \). \hfill [4 marks]
			
			\item[(ii)] Write the observer equations. \hfill [3 marks]
			
			\item[(iii)] Discuss the trade-off in selecting observer pole locations. \hfill [2 marks]
		\end{enumerate}
	\end{enumerate}











\end{document}





