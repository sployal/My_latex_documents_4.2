\documentclass[12pt]{article}
\usepackage{graphicx} % Required for inserting images
\usepackage{amsmath}
\usepackage{newtxtext,newtxmath} % Times New Roman font for text and math
\usepackage[a4paper, margin=1in]{geometry} % Adjust margins to resemble Word
\usepackage{caption} % For adding a caption without the figure environment
\usepackage{ragged2e} % For text justification
\usepackage{booktabs}

\title{CAT 1: Electrical machine design} % Correct the title
\author{David Muigai} % Ensure the author name is correct



\begin{document}
	
	\maketitle


\section*{Electrical Engineering Questions}

\subsection*{\textbf{1.} Enumerate four desirable characteristics of a conducting material that can be used in electrical machine design. \textbf{(4 marks)}}
\begin{itemize}
	\item \textbf{High Electrical Conductivity:} This minimizes resistive losses and improves efficiency by allowing current to flow with minimal resistance.
	\item \textbf{High Thermal Conductivity:} This helps in dissipating heat generated during operation, preventing overheating and enhancing the longevity of the machine.
	\item \textbf{Mechanical Strength and Durability:} The material should be strong enough to withstand mechanical stresses and vibrations during operation without degrading.
	\item \textbf{Corrosion Resistance:} The material should resist oxidation and other forms of corrosion to maintain conductivity and structural integrity over time.
	\item \textbf{Low Cost and Availability:} The material should be affordable and readily available to ensure cost-effective manufacturing.
	\item \textbf{Lightweight:} A lower density material reduces the overall weight of the machine, which is essential for applications like electric vehicles and aerospace systems.
	\item \textbf{Flexibility and Malleability:} This allows easy shaping and winding, especially in the case of wire conductors.
\end{itemize}


\subsection*{\textbf{2.} What are the advantages and disadvantages of large air gap length in induction motor? \textbf{(2 marks)}}





\subsection*{\textbf{3.} Using neat diagrams, differentiate between Oil natural-air forced (ONAF) and Oil forced-air natural (OFAN) methods of transformer cooling. \textbf{(4 marks)}}

\subsection*{\textbf{4.} The ratio of flux to full load mmf in a 400kVA, 50Hz, single phase core type transformer is $2.4 \times 10^{-6}$. Calculate the net iron area and the window area of the transformer if the maximum flux density in the core is $1.3$ Wb/m$^2$, current density $2.7$ A/mm$^2$ and window space factor is $0.26$. Also, calculate the full load mmf. \textbf{(7 marks)}}

\subsection*{\textbf{5.} Find the main dimensions of a 2500kVA, 187.5 rpm, 50Hz, 3-phase, 3kV, salient pole synchronous generator. The generator is to be vertical, water wheel type. The specific magnetic loading is $0.6$ Wb/m$^2$ and the specific loading is $34000$ A/m. Use circular poles with a ratio of core length to pole pitch $= 0.65$. Assume a winding factor of $0.955$. \textbf{(7 marks)}}

\subsection*{\textbf{6.} What is window space factor $K_w$ as used in electrical machine design? \textbf{(2 marks)}}

\subsection*{\textbf{7.} Design a DC shunt generator rated at 100kW, 220V, 4-pole, operating at 1500 rpm with a square pole face. Given: Specific magnetic loading, $B_{av} = 0.40$ Wb/m$^2$, Specific electric loading, $q = 14000$ A/m, Pole arc to pole pitch ratio, $= 0.65$, Full-load efficiency, $\eta = 0.90$. Determine the main dimensions (armature diameter $D$ and effective core length). \textbf{(6 marks)}}




\end{document}
