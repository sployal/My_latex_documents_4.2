\documentclass[12pt]{article}
\usepackage{graphicx} % Required for inserting images
\usepackage{amsmath}
\usepackage{newtxtext,newtxmath} % Times New Roman font
\usepackage[a4paper, margin=1in]{geometry} % Standard margins
\usepackage{ragged2e} % For text justification
\usepackage{titlesec} % For title formatting
\usepackage{setspace} % Line spacing
\usepackage{array} % Table alignment


\begin{document}
	
	% University Logo
	\begin{center}
		\includegraphics[width=6cm]{images/picture1} % Replace with the actual logo filename
	\end{center}
	
	% Institution and Department
	\begin{center}
		\textbf{\large CHUKA UNIVERSITY} \\[1.5cm]
		\textbf{FACULTY OF ENGINEERING AND TECHNOLOGY} \\[0.3cm]
		\textbf{DEPARTMENT OF ELECTRICAL AND ELECTRONIC ENGINEERING} \\[2cm]
	\end{center}
	
	% Assignment Title
	\begin{center}
		\textbf{\large EENG 465 Power Systems II} \\[0.3cm]
		\textbf {\large \textcolor{teal}{CAT 1 }} \\[1cm] 
	\end{center}
	
	% Group Members Section
	\noindent
	\textbf{\fontsize{14}{16} \selectfont Group Members} \\[0.4cm] % You can adjust the size
	
	\begin{tabular}{l@{\hspace{3cm}}l} 
		\fontsize{15}{15} \selectfont Johnson Kiama  & \fontsize{13}{15} \selectfont EB24/56163/21  \\[0.2cm]
		\fontsize{15}{15} \selectfont Mary Wangari   & \fontsize{13}{15} \selectfont EB24/56151/21  \\[0.2cm]
		\fontsize{15}{15} \selectfont David Muigai   & \fontsize{13}{15} \selectfont EB24/56171/21  \\[0.2cm]
		\fontsize{15}{15} \selectfont Johnson Munyasia & \fontsize{13}{15} \selectfont EB24/56157/21  \\[0.2cm]
		\fontsize{15}{15} \selectfont Elijah Karanja & \fontsize{13}{15} \selectfont EB24/56201/21  \\[0.2cm]
	\end{tabular}
	
	
	
	\newpage 
	
	
	\section*{Question 1}
	
	
	
	A 275 kV overhead transmission line has the following characteristics:  
	\( Z = 12.5 + j66 \, \Omega \), \( Y = j4.4 \times 10^{-4} \, S \) 
	The line delivers 250 MW at a lagging power factor of 0.9. Determine:
	
	\begin{enumerate}
		\item[\textbf{(i)}] ABCD constants
		\item[\textbf{(ii)}] Surge impedance of the line
		\item[\textbf{(iii)}] Sending-end voltage
		\item[\textbf{(iv)}] Sending-end current
		\item[\textbf{(v)}] Line charging current
		\item[\textbf{(vi)}] Transmission efficiency
		\item[\textbf{(vii)}] Voltage regulation
	\end{enumerate}
	
	
	\subsection*{Solution} 
	
	\textbf{Given:}  
	\[
	Z = 12.5 + j66, \quad Y = 4.4 \times 10^{-4} \angle 90^\circ
	\]
	\[
	P = 250 \text{MW}, \quad V_R = 275 \text{kV}, \quad \cos\phi_R = 0.9
	\]
	\[
	\phi_R = \cos^{-1} 0.9 = 25.842^\circ
	\]
	
	
     \textbf{The receiving end active power is given by ,}
	\begin{equation}
		P = \sqrt{3} V_R I_R \cos\phi_R
	\end{equation}
	
	\begin{equation}
		I_R = \frac{250 \times 10^6}{\sqrt{3} \times 275 \times 10^3 \times 0.9} = 583.18 A
	\end{equation}
	
	\[
	I_R = 583.18 \angle -25.842^\circ A
	\]
	
	\renewcommand{\labelenumi}{(\textbf{\roman{enumi}})}
	\begin{enumerate}
	\item \textbf{Calculating ABCD constant for long transmission line,} 
	
	\begin{equation}
		A = D = \cosh \gamma = \frac{1}{2} \left[ e^{\alpha + j\beta} + e^{-(\alpha + j\beta)} \right]
	\end{equation}
	
	\[ A = D = \frac{1}{2} \left[ e^{\alpha} \angle \beta + e^{-\alpha} \angle -\beta \right] \]
	
	\begin{equation}
		\gamma = \sqrt{ZY}
	\end{equation}
	
	\[ \gamma = \sqrt{(12.5 + j66) \times (4.4 \times 10^{-4} \angle 90^\circ)} \]
	
	\[ \gamma = \sqrt{0.0295 \angle 169.27^\circ} \]
	
	\begin{equation}
		\gamma = 0.172 \angle 84.635^\circ = 0.0160 + j0.171
	\end{equation}
	
	
	
	
	\textbf{comparing with $\alpha + j\beta$, we get the following}
	
	\[ \alpha = 0.0160 \text{ rad}, \quad \beta = \frac{0.171 \times 180}{\pi} = 9.797^\circ \]
	
	\begin{equation}
		A = D = \frac{1}{2} \left[ e^{0.016} \angle 9.797^\circ + e^{-0.016} \angle -9.797^\circ \right]
	\end{equation}
	
	\begin{equation}
		A = D = 0.9855 \angle 0.1582^\circ
	\end{equation}
	
	\begin{equation}
		\sinh \gamma = \frac{1}{2} \left[ e^{\alpha} \angle \beta - e^{-\alpha} \angle -\beta \right]
	\end{equation}
	
	\[ = \frac{1}{2} \left[ e^{0.016} \angle 9.797^\circ - e^{-0.016} \angle -9.797^\circ \right] \]
	
	\begin{equation}
		\sinh \gamma = 0.171 \angle 84.71^\circ
	\end{equation}
	

	\item \textbf{The surge impedance of the line (\(Z_0\)) is given as, }
	
	\begin{equation}
		Z_0 = \sqrt{\frac{Z}{Y}}
	\end{equation}
	
	Substituting the given values:
	
	
	\[Z_0 = \sqrt{\frac{12.5 + j66}{j4.4 \times 10^{-4}}} \]
	
	
	\[Z_0 = \sqrt{152666.55 \angle -10.724^\circ} \]
	
	\begin{equation}
		Z_0 = 390.725 \angle -5.362^\circ \, \Omega
	\end{equation}
	
	\text{so that}
	
	\begin{equation}
		B = Z_0 \sinh \gamma
	\end{equation}
	
	\[ B = 390.725 \angle -5.362^\circ \times 0.171 \angle 84.71^\circ \]
	
	\begin{equation}
		B = 66.814 \angle 79.348^\circ
	\end{equation}
	
	\begin{equation}
		C = \frac{\sinh \gamma}{Z_0} = \frac{0.171 \angle 84.71^\circ}{390.725 \angle -5.36^\circ}
	\end{equation}
	
	\begin{equation}
		C = 4.376 \times 10^{-4} \angle 90.072^\circ
	\end{equation}
	
	
	
	
	
	\item \textbf{Sending end voltage is given as,}
	
	\begin{equation}
		V_s = A V_R + B I_R
	\end{equation}
	
	\[
	V_s = (0.9855 \angle 0.1582^\circ) \times \left( \frac{275 \times 10^3}{\sqrt{3}} \right)
	\]
	
	\[
	+ (66.814 \angle 79.348^\circ) \times (583.18 \angle -25.842^\circ)
	\]
	
	\begin{equation}
		V_s = 182427.57 \angle 0.025^\circ V
	\end{equation}
	
	
	\item \textbf{Sending end current is given as, }
	
	\begin{equation}
		I_s = C V_R + D I_R
	\end{equation}
	
	
	\[  I_s = \left( 4.376 \times 10^{-4} \angle 90.072^\circ \right) \left( \frac{275 \times 10^3}{\sqrt{3}} \angle 0^\circ \right)   \]
	
	
	\[+ \left( 0.9855 \angle 0.1582^\circ \right) (583.18 \angle -25.842^\circ) \]	
	
	\[ I_s = 69.478 \angle 90.072^\circ + 574.72 \angle -25.68^\circ \]
	\begin{equation}
		I_s = 548.12 \angle -19.12^\circ \text{ A} \quad 
	\end{equation}
	
	\item \textbf{The line charging current is given by:} 
	
	\begin{equation}
		I_c = j\omega C V_R
	\end{equation}
	
	From the ABCD calculations:
	

	\[C = 4.376 \times 10^{-4} \angle 90.072^\circ \]
	
	
	\[I_c = \left( 4.376 \times 10^{-4} \angle 90.072^\circ \right) \times \left( \frac{275 \times 10^3}{\sqrt{3}} \right) \]
	
	\begin{equation}
		I_c = 69.478 \angle 90.072^\circ \, A
	\end{equation}
	
	
	\textbf{Sending end power factor,}
	
	\begin{equation}
		\phi_s = 0.025^\circ - (-19.12^\circ) = 19.145^\circ
	\end{equation}
	
	\begin{equation}
		\cos\phi_s = \cos 19.145^\circ = 0.9448
	\end{equation}
	
	
	
	\item \textbf{Calculating the transmission efficiency of the  line,}
	
	\begin{equation}
		\eta = \frac{\text{Output}}{\text{Input}} = \frac{3 V_R I_R \cos \phi_R}{3 V_S I_S \cos \phi_S}
	\end{equation}
	
	
	\[ \eta = \frac{250 \times 10^6}{3 \times 182427.57 \times 548.12 \times 0.8733}  \]
	
	\begin{equation}
		\eta = 95.43\%
	\end{equation}
	
	
	
	\item \textbf{Finally the Voltage  regulation of transmission line,}
	
	\begin{equation}
		\text{Regulation} = \frac{\left| \frac{V_S}{A} \right| - |V_R|}{|V_R|}
	\end{equation}
	
	
	\[ = \frac{\frac{182427.57}{0.9855} - \frac{275 \times 10^3}{\sqrt{3}}}{\frac{275 \times 10^3}{\sqrt{3}}}  \]
	
	\begin{equation}
		= 0.1659 = 16.59\%
	\end{equation}
	
    \end{enumerate}

	
	
	\noindent
	2. (15 pts.) Consider the following system:
	\[
	\text{[Diagram not shown]}
	\]
	
	Here, the system \( F \) is defined by the input-output relationship
	\[
	F\{z[n]\} = z[n] - z[n-1],
	\]
	and \( \Delta \) is the unit delay
	\[
	\Delta\{w[n]\} = w[n-1].
	\]
	Write down the linear difference equation describing this system.
	
	\textbf{Solution.} Let \( z[n] \) be the output of the summer, as shown above. Then
	\[
	y[n] = F\{z[n]\} = z[n] - z[n-1].
	\]
	
	Now,
	\[
	z[n] = 2x[n] - \Delta\{y[n]\} = 2x[n] - y[n-1].
	\]
	Therefore, substituting the expression for \( z[n] \) into the first equation, we can write
	\[
	\begin{aligned}
		y[n] &= z[n] - z[n-1] \\
		&= \underbrace{2x[n] - y[n-1]}_{=z[n]} - \underbrace{2x[n-1] - y[n-2]}_{=z[n-1]} \\
		&= 2x[n] - y[n-1] - 2x[n-1] + y[n-2].
	\end{aligned}
	\]
	
	\[
	\fbox{y[n] + y[n-1] - y[n-2] = 2x[n] - 2x[n-1]}
	\]
	
	
	
	
	

\end{document}
